\documentclass{article}
\usepackage{summaries}

%Setup:%
\newcommand{\doctitle}{Klagearten}
\newcommand{\topic}{Verwaltungsrecht AT}
\newcommand{\license}{Dieses Dokument von Keanu Dölle steht unter CC BY 4.0}

\begin{document}

\section{Übersicht Klagearten}
\subsection{Anfechtungsklage}
\begin{definition}{Anfechtungsklage}
Bei einer Anfechtungsklage wird die Aufhebung eines Verwaltungsaktes begehrt.
\end{definition}

\begin{important-box}
Das ist etwas ganz ganz \textbf{Wichtiges}

Lorem ipsum dolor sit amet, consectetur adipiscing elit, sed do eiusmod tempor incididunt ut labore et dolore magna aliqua. Ut enim ad minim veniam, quis nostrud exercitation ullamco laboris nisi ut aliquip ex ea commodo consequat.
\end{important-box}

\begin{legal-text}{§ 62 StGB}
(1) Test123Lorem ipsum dolor sit amet, consectetur adipiscing elit, sed do eiusmod tempor incididunt ut labore et dolore magna aliqua. Lorem ipsum dolor sit amet, consectetur adipiscing elit, sed do eiusmod tempor incididunt ut labore et dolore magna aliqua.\\

(2) Lorem ipsum dolor sit amet, consectetur adipiscing elit, sed do eiusmod tempor incididunt ut labore et dolore magna aliqua. 
\end{legal-text}

\begin{schemata}{Anfechtungsklage}
\begin{enumerate}[\textbf{A.}]
   \item \textbf{First level item}
   \item \textbf{First level item}
   \begin{enumerate}[I.]
     \item Second level item
     \item Second level item
     \begin{enumerate}[1.]
       \item Third level item
       \item Third level item
       \begin{enumerate}[a)]
         \item Fourth level item
         \item Fourth level item
       \end{enumerate}
     \end{enumerate}
   \end{enumerate}
 \end{enumerate}
\end{schemata}
\subsection{Verpflichtungsklage}
\subsection{allgemeine Leistungsklage}
\subsection{Feststellungsklage}
\subsection{Fortsetzungsfeststellungsklage}
\subsection{Normenkontrollverfahren}
\end{document}