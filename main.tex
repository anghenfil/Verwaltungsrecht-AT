\documentclass{article}
\usepackage{summaries}

%Setup:%
\newcommand{\doctitle}{Klagearten}
\newcommand{\topic}{Verwaltungsrecht AT}
\newcommand{\license}{Dieses Dokument von Keanu Dölle steht unter CC BY 4.0}

\begin{document}

\section{Übersicht Klagearten}
\subsection{Anfechtungsklage}
\begin{definition}{Anfechtungsklage}
Bei einer Anfechtungsklage wird die Aufhebung eines Verwaltungsaktes begehrt.
\end{definition}

Umstritten ist, ob auch \textbf{Nebenbestimmungen} einzeln mittels Anfechtungsklage angefochten werden können. Heute wird dies überwiegend bejaht, die Möglichkeit einer Teilanfechtung folge schon aus § 113 I 1 VwGO. Geregelt wird die Anfechtungsklage in § 42 I 1. Alt. VwGO:
\begin{legal-text}{§ 42 I 1 . Alt VwGO}
(1) \textbf{Durch Klage kann die Aufhebung eines Verwaltungsakts (Anfechtungsklage)} sowie die Verurteilung zum Erlaß eines abgelehnten oder unterlassenen Verwaltungsakts (Verpflichtungsklage) begehrt werden.
[...]
\end{legal-text}

\begin{schemata}{Anfechtungsklage}
\begin{enumerate}[\textbf{A.}]
   \item \textbf{Zulässigkeit der Anfechtungsklage}
   \begin{enumerate}[I.]
        \item ggf. Deutsche Gerichtsbarkeit (§§ 18, 19 GVG analog)
        \item Eröffnung des Verwaltungsrechtswegs (§ 40 I VwGO)
        \begin{enumerate}[1.]
            \item öffentlich-rechtliche Streitigkeit
            \item nichtverfassungsrechtlicher Art
            \item keine abweisende Sonderzuweisung
        \end{enumerate}
        \item Statthafter Rechtsbehelf
        \begin{enumerate}[1.]
            \item Rechtsschutzziel
            \begin{enumerate}
                \item Rechtscharakter der anzugreifenden Regelung
                \item ggf. Nebenbestimmung (s.O.)
            \end{enumerate}
            \item Entsprechender Rechtsbehelf
        \end{enumerate}
        \item Zuständigkeit des Verwaltungsgerichts
        \begin{enumerate}[1.]
            \item örtliche Zuständigkeit (§ 52 VwGO)
            \item sachliche Zuständigkeit (§ 45 VwGO)
        \end{enumerate}
        \item Beteiligtenfähigkeit (§ 61 VwGO)
        \item Prozessfähigkeit (§ 62 VwGO) / ordnungsgemäße Prozessvertretung (§§ 67, 67a VwGO)
        \item Klagebefugnis (§ 42 II VwGO)
        \item Vorverfahren (§ 68 VwGO)
        \item Klagefrist (§ 74 I VwGO)
        \item Ordnungsmäßigkeit der Klageerhebung (§§ 81, 82 VwGO)
        \item Allgemeines Rechtschutzbedürfnis
   \end{enumerate}
   \item \textbf{Begründetheit der Anfechtungsklage}
 \end{enumerate}
\end{schemata}

Test: \ltext*{VwGO}{52}{1. Var.}

Test: \ltext{VwGO}{52}{1. Var.}
\subsection{Verpflichtungsklage}
\subsection{allgemeine Leistungsklage}
\subsection{Feststellungsklage}
\subsection{Fortsetzungsfeststellungsklage}
\subsection{Normenkontrollverfahren}
\end{document}